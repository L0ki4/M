\section{Минимальное свойство коэффициентов Фурье по ортогональной системе. Неравенства Бесселя.}
$ \mathcal{E} $ --- предгильбертово пространство\\
$ \Phi = \{ \varphi_1, \varphi_2, \ldots, \varphi_n,\ldots \} \subset \mathcal{E}$

\begin{greyDefinition}
	Система $ \Phi $ называется \textbf{ортогональной}, если $ \forall i,j: i\neq j \mapsto (\varphi_i,\varphi_j) =0 $. Если при этом $ \forall i \mapsto (\varphi_i,\varphi_i) =1,  $ то система $ \Phi $ называется \textbf{ортонормированной}. 
\end{greyDefinition}

Начальные условия: $ \mathcal{E} $ --- предгильбертово пространство, $ \Phi = \{ \varphi_1,\varphi_2,\ldots,\varphi_n,\ldots \} $ --- счетная ортонормированная система элементов из $ \mathcal{E} $
\begin{greyDefinition} Назовем \textbf{рядом Фурье} элементов $ f \in \mathcal{E} $ по системе $ \Phi $ следующий ряд
	\[
	\sum_{k=1}^{\infty} f_k\varphi_k,\; f_k = (f,\varphi_k)
	\]
	$ f_k $ --- \textbf{коэффициент Фурье} элемента $ f $
	
	\textbf{Частичная сумма ряда Фурье} элемента $ f $ по системе $ \Phi $
	\[
	S_n^f = \sum_{k=1}^n f_k\varphi_k,\; f_k = (f,\varphi_k)
	\]
	
	$||f-g|| $ --- \textbf{отклонение элемента} $ g $ от элемента $ f $. 
	
	$ ||f||=\sqrt{(f,f)} $ --- \textbf{норма} элемента $ f $
\end{greyDefinition}
\begin{greyTheorem}\textbf{Основная теорема}.	Из всех сумм вида $ \sum_{k=1}^{n} c_k \varphi_k $, наименьшее отклонение по норме данного пространства от элемента $ f $ имеет частичная сумма ряда Фурье элемента $ f $ по $ \Phi $
\end{greyTheorem}
\begin{greyProof}
	\[
	||f-\sum_{k=1}^n c_k\varphi_k ||^2 = (f-\sum_{k=1}^n c_k\varphi_k,\; f -\sum_{k=1}^n c_k\varphi_k ) = \sum_{k=1}^{n}c_k^2 - 2\sum_{k=1}^n c_k(f,\varphi_k) + ||f||^2
	\]
	выделим полный квадрат и учтем, что $ (f,\varphi_k) = f_k$
	\[
	||f-\sum_{k=1}^n c_k\varphi_k ||^2 = \sum_{k=1}^n(c_k-f_k)^2 + ||f||^2 - \sum_{k=1}^n f_k^2 	
	\]
	Следовательно отклонение наименьшее, если $ c_k = f_k $
\end{greyProof}
%Важное следствие
\paragraph{Следствие 1.} Для любого элемента $ f \in \mathcal{E} $, для любых чисел $ c_k $, для любой счетной ортонормированной системы элементов $ \Phi  $ и для любого фиксированного числа $ n $, выполняется неравенство:
\[
||f||^2 - \sum_{k=1}^n f_k^2 \leqslant ||f-\sum_{k=1}^n c_k\varphi_k||^2
\]
\begin{greySmth}{Следствие 2.}Для любого элемента $ f \in \mathcal{E} $, для любой счетной ортонормированной системы элементов $ \Phi  $ и для любого фиксированного числа $ n $, справедливо \textbf{тождество Бесселя}:
\[
||f||^2 -\sum_{k=1}^n f_k^2 = ||f-\sum_{k=1}^n f_k\varphi_k||^2
\]
\end{greySmth}
\begin{greySmth}{Следствие 3.} Для любого элемента $ f \in \mathcal{E} $, для любой счетной ортонормированной системы элементов $ \Phi  $ и для любого фиксированного числа $ n $, справедливо \textbf{неравенство Бесселя}:
\[
\sum_{k=1}^\infty f_k^2 \leqslant ||f||^2
\]
\end{greySmth}
\begin{greyProof}
	Из тождества Бесселя:
	\[
	\sum_{k=1}^n f_k^2 \leqslant ||f||^2
	\]
	$ S_n = \sum_{k=1}^n f_k^2 $ --- монотонно возрастающая последовательность, к тому же она ограниченна из неравенства выше, следовательно $ \sum_{k=1}^{\infty} f_k^2 < \infty $. 
	%перейдя к пределу при n стремящимся к бесконечности и воспользовавшись предельным переходом в неравествах можно доказать.
\end{greyProof}